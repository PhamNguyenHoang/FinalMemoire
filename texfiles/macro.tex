% !TEX encoding = UTF-8 Unicode

%%%%%%%%%%%%%%%%%%%%%%%%%%%%%%%%%%%%%%%%%%%%%%%%%%%%%%%
%% EN-TETES ET PIEDS DE PAGE
\usepackage{fancyhdr}
\pagestyle{fancy}% pour activer le style de pages personnalisÈ
\fancyhf{}%remise ‡ zÈro des en-tÍte et pied de page
\setlength{\headheight}{14pt} % pour fixer la hauteur de l'espace rÈservÈ ‡ l'en-tÍte du haut

%%% Pas de numÈro de page sur la premiËre page des chapitres
\makeatletter
\let\ps@plain=\ps@empty
\makeatother

%===================== Style 1 =================================================
%En-tÍte : 
% * dans la boite de droite (R), pour les pages impaires (O)
% * et dans la boite de gauche (L), pour les pages paires (E)
% mettre le numÈro de page (\thepage).
\fancyhead[RO,LE]{% 
\thepage
}
\fancyhead[LO]{\scshape \nouppercase{\rightmark}}  %%%Section
\fancyhead[RE]{\scshape \nouppercase{\leftmark}} %%% Chapitre 
\renewcommand{\headrulewidth}{.4pt}
\fancyfoot{}


%================================== Style 2 ====================================

% \fancyfoot[RO,LE]{% Boite de droite (R), pages impaires(O) et Boites de gauche pages paires
% \thepage
% }
% \fancyhead[CO]{\slshape \nouppercase{\rightmark}}  %%%Section
% \fancyhead[CE]{\slshape \nouppercase{\leftmark}} %%% Chapitre 
% \renewcommand{\headrulewidth}{.4pt}

% Remarques generales :
% nouppercase permet l'affichage en minuscules au lieu de majuscules
% slshape permet l'affichage en lettres penchÈs
% scshape permet l'affichage en petites capitales

% Pour que les pages paires sans texte (par exemple, ‡ la fin d'un chapitre et
% avant un autre), ne contiennent ni en-tÍte ni pied de page (source :
% http://www.tex.ac.uk/cgi-bin/texfaq2html?label=reallyblank)
\let\origdoublepage\cleardoublepage
\newcommand{\clearemptydoublepage}{%
  \clearpage
  {\pagestyle{empty}\origdoublepage}%
}
\let\cleardoublepage\clearemptydoublepage

% RÈglage fin des notes de bas de page
\FrenchFootnotes % pour les notes de bas de page ‡ la franÁaise
\AddThinSpaceBeforeFootnotes % pour avoir une espace fine entre le mot et l'appel de note


%%%%%%%%%%%%%%%%%%%%%%%%%%%%%%%%%%%%%%%%%%%%%%%%%%%%%%%
%% CHAPITRE ETOILE
%% avec rÈfÈrence dans la table des matiËres et les bons en-tÍtes
%% il sert pour l'introduction, la page de notations.
\newcommand*\chapterstar[1]{%
  \chapter*{#1}%
  \addcontentsline{toc}{chapter}{#1}%
  \markboth{#1}{#1}}

%%%%%%%%%%%%%%%%%%%%%%%%%%%%%%%%%%%%%%%%%%%%%%%%%%%%%%%
% ENVIRONNEMENTS DE THEOREMES
\theoremstyle{plain} % style plain
\newtheorem{theo}{Théorème}[chapter]
\newtheorem{cor}[theo]{Corollaire}
\newtheorem{prop}[theo]{Proposition}
\newtheorem{lem}[theo]{Lemme}
\newtheorem{conj}[theo]{Conjecture}
\newtheorem*{theoetoile}{Théorème} % thÈorËme non numÈrotÈ
\newtheorem*{conjetoile}{Conjecture} % conjecture non numÈrotÈe

\theoremstyle{definition} % style definition
\newtheorem{defi}[theo]{Définition}
\newtheorem{exemple}[theo]{Exemple}
\newtheorem{question}[theo]{Question}
\newtheorem{remarque}[theo]{Remarque}
\newtheorem{notation}[theo]{Notation}

% Pour renommer ``preuve'' en ``dÈmonstration''
\renewcommand{\proofname}{Démonstration}


%%%%%%%%%%%%%%%%%%%%%%%%%%%%%%%%%%%%%%%%%%%%%%%%%%%%%%%
% ENVIRONNEMENTS DEDICACE ET EPIGRAPHE
\newenvironment{dedicace}{%
  \newpage\thispagestyle{empty}
  \hfill\begin{minipage}{100mm}\begin{flushright}\it}{%
  \end{flushright}\end{minipage}\vfill}

\newenvironment{epigraphe}{%
  \hfill\begin{minipage}{60mm}\begin{flushright}\footnotesize\it}{%
  \end{flushright}\end{minipage}\hspace*{7mm}\vfill}
  
  
%\NeedsTeXFormat{LaTeX2e}
%\newif\if@sommairechap \@sommairechapfalse
%\DeclareOption{sommairechap}{\@sommairechaptrue}
%\ProcessOptions

 % Gestion Èventuel des sommaires en dÈbut de chapitre
%\if@sommairechap
%  \RequirePackage[french]{minitoc}
%  \dominitoc
%  \setlength{\mtcindent}{0em}
%  \renewcommand{\mtifont}{\large\rm\scshape}
%  \renewcommand{\mtcSfont}{\small\rm\scshape}
%\if
%\RequirePackage{tocloft}
%\renewcommand{\cftsubsecfont}{\small}
%\renewcommand{\cftsecfont}{\normalsize\scshape}
%\renewcommand{\cftchapfont}{\large\scshape}
%\renewcommand{\cfttoctitlefont}{\Huge\scshape}
%\renewcommand{\cftloftitlefont}{\Huge\scshape}
%
%
%
%% ==========================================================================
%% PAGE DE GARDE DES CHAPITRES
%% Le rÈsumÈ + le minitoc Èventuel
%\newenvironment{chapintro}{%
%  \if@sommairechap \nomtcrule \vspace{1.5cm} \minitoc[l] \fi}{%
%  \cleardoublepage
%}
% 
%% chapitre numÈrotÈ
%\newfont{\chapterNumber}{eurb10 scaled 7000}
%\renewcommand*{\@makechapterhead}[1]{%
%  \thispagestyle{plain}
%  \marginpar{\vspace*{1.5em}\flushright\chapterNumber\thechapter}
%  \begin{flushleft}\nobreak\Huge\sc#1\end{flushleft}
%  \vspace{3cm}
%}
%% chapitre non numÈrotÈ (*) 
%\renewcommand*{\@makeschapterhead}[1]{%  
%  \markboth{#1}{#1}
%  \thispagestyle{plain}
%  \begin{flushleft}\nobreak\Huge\sc #1\end{flushleft}
%  \vspace{3cm}
%  \if@sommairechap \mtcaddchapter \fi
%}
