% !TEX encoding = UTF-8 Unicode
\pdfbookmark[0]{Résumé}{resume}
    
\chapter*{Résumé}

%\begin{center} \Large \bf Thèse intitulée: Séparation des Préoccupations en Épidémiologie \end{center}
La dengue hémorragique (DH) est une maladie à transmission vectorielle très répandue dans les climats tropicaux et subtropicaux du monde, principalement dans les zones urbaines et semi-urbaines. Selon l'Organisation mondiale de la santé (OMS, 2007), environ 40\% de la population mondiale dans 112 pays du monde est exposée au virus. On estime à 390 millions le nombre de nouvelles infections chaque année. Le problème d’identification de la relation entre cette maladie et l’influence des facteurs environnementaux et climatiques existe depuis longtemps. Cependant, les influences réelles exactes des variables climatiques sur la transmission des maladies transmises par les moustiques restent incertaines dans la plupart des cas attendus.

Dans cette thèse, nous analysons des données temporelles incluant le nombre de cas de dengue collectés mensuellement dans 273 provinces de huit pays de l’Asie du Sud-Est: Thaïlande, Cambodge, Laos, Vietnam, Malaisie, Indonésie, Philippines et Taiwan. Ces données couvrent une zone géographique de 3 500 kilomètres d'est en ouest et de 2 500 kilomètres du nord au sud. Au total, elles comptaient 320 millions d'habitants en 2010. Avant d'analyser les données, nous effectuons quelques opérations de pré-traitement. Puis, nous avons construit une vidéo qui décrit le taux d’infection de chaque ville à partir 1994 au 2010. Ensuite, nous appliquons une méthode de clustering k-means pour classifier 273 provinces en fonction de leurs taux d’infection. La distance entre les séries temporelles de leurs provinces est calculée à l'aide de la méthode Dynamic Time Wrapping (DWT). La méthode d'analyse silhouette a été utilisé pour évaluer le nombre des cluster de la méthode k-means.

Nous avons ensuite effectué quelques méthode pour analyser la relation entre les facteurs climatiques et l’épidémie de dengue dans 64 provinces du pays Vietnam, où les données sont plus complètes. Les résultats des algorithmes montrent une fort effet de la température et de l'humidité absolue sur l'incidence de la dengue. Cependant, il existe toujours un effet indirect des précipitations, de l'humidité relative et des heures d'ensoleillement sur la dynamique de l'épidémie de dengue lorsque leur combinaison donne un meilleur résultat que leur analyse uni-variable.
 
Notre étude présente un aperçu de l’influence des facteurs environnementaux sur l’évolution de l’épidémie de dengue. À l'avenir, nous devons approfondir l'analyse des données des provinces d'autres pays de la région de l'Asie du Sud-Est afin d'obtenir un résultat plus complet sur ce problème.
\subsubsection*{Mots clefs : dengue épidémie, modèles VAR, k-means clustering, Granger Causality Analysis, analyse ondelette, modèle EDM, Convergence Cross-Mapping, distance entre les séries temporelles}



%\pagebreak
\begin{otherlanguage}{english}

\vspace{1cm}
\begin{center} \rule{\textwidth/3}{1pt} \end{center}
\vspace{1cm}

%\begin{center} \Large \bf Separation of concerns in epidemiologie \end{center}
\section*{Abstract}
Dengue hemorrhagic fever (DHF) is a vector-borne disease that is very common in tropical and sub-tropical climates worldwide, mostly in urban and semi-urban areas. According to the World Health Organization (WHO, 2007), about 40\% of the world's population in 112 countries of the world is exposed to the virus. An estimated 390 million new infections every year. The problem of identifying the relation between this disease and the influence of environmental and climatic factors has existed for a long time. However, the exact real influences of climatic variables on the transmission of mosquito-borne diseases remain uncertain in most of the expected cases.

In this thesis, we perform some analysis on time series data that included the number of dengue case collected monthly from 273 provinces in eight Southeast Asian countries : Thailand, Cambodia, Laos, Vietnam, Malaysia, Indonesia, Philippines and Taiwan. This data covers a geographical area of 3,500 kilometers from east to west over 2,500 kilometers from north to south and had a combined population of 320 million in 2010. Before analyzing the data, we perform some preprocessing steps on the data. Then we construct a video describing the infection's rate of each city from 1994 to 2010. After that, we perform a k-means clustering method to classified 273 provinces by their infection's rate. The distance between the time series of their provinces are calculated by the dynamic time wrapping (DWT) method. The silhouette analysis method was used to evaluate the number of cluster of the k-means method.

Then, we performed some analytical method to analyse the relationship between climatic factors and the dengue epidemic in 64 provinces of Vietnam Country, where data are more complete. The results of their algorithm show the strong effect of the temperature and absolute humidity on the dengue incidence. However, there still exist an indirect effect of rainfall, relative humidity and hours of sunshine on the dynamique of dengue epidemic when their combination gives better result than their uni-variable analysis.

Our study show an overview about the influence of environmental factors ont the evolution of the dengue epidemic. In the future, we need to deepen data analysis of provinces from other countries in the Southeast Asia region to obtain a more complete an detailed result on this problem. 

\subsection*{Keywords: dengue epidemic, VAR models, k-means clustering, Granger causality analysis, wavelet analysis, EDM model, cross convergence mapping, distance between time series}

\end{otherlanguage}   
