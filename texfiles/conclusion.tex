% !TEX encoding = UTF-8 Unicode

\chapter{Conclusion}
\label{conclusion}

\section{Résultats de la thèse}
Nous avons présenté dans cette thèse nos contribution pour le problème de détection les relations entre les facteurs climatiques et la maladie infectieuse en utilisant les données réels des provinces des pays en Asie du Sud-Est. 

La structure et les caractéristiques des données sont été présenté dans le chapitre 2. Afin de faciliter le traitement des analyse, nous avons effectué des étape de pré-traitement sur les données comme l'interpolation, la normalisation et le retranchement sur les séries temporelles. Dans le chapitre 3, nous avons appliqué la méthode k-means pour classifier les taux d'inflections de 273 provinces au 9 pays du région Sud-Est d'Asie. Les taux d'infection des provinces sont classifié dans les régions qui ont des situation géographique proche les uns des autres.  Ces régions peuvent être affectés par des même conditions environnementales, ce qui peut conduire à une similarité des taux d'infection.  

Dans les chapitre 4, 5 et 6, nous avons utilisés les différentes méthodes basé sur les différentes contexte pour effectuer l'analyse sur la relation entre le taux d'infection et les facteurs environnementaux du 64 provinces du Vietnam à partir Janvier 1998 jusqu'au Septembre 2010. Les résultats de ces méthode ont montrés des même conclusion :
\begin{itemize}
\item[$\bullet$] La température et l'humidité absolue sont fortement affecté sur l'influence de la dengue dans plupart des provinces du Vietnam.
\item[$\bullet$] La pluviosité, l'humidité relative et l'heure d'ensoleillement ont un impact faible sur l'influence de la dengue dans certaines provinces du Vietnam.
\item[$\bullet$] Dans la plupart des provinces, la dengue est plus répandue pendant la saison des pluies, mais à l'échelle interannuelle, les épidémies de dengue ont également été associées à la sécheresse. 
\end{itemize}

Cependant, Ils existent des différentes avantages des algorithmes : 
\begin{itemize}
\item[$\bullet$] La méthode analyse ondelette dans le chapitre 4 permet de trouver les corrélation positive et négative dans les cycles spécifique de la maladie. 
\item[$\bullet$] La méthode GCA univariable permet de trouver les relation directement entre les facteurs climatiques et l'incidence de la dengue. De plus, les résultats du GCA multivariable montre des implications  latentes entre des facteurs climatiques dans la mesure où ils contribuent ensemble à l'épidémie de la dengue.
\item[$\bullet$] Le modèle EDM a montré une relation non-linéaire entre l'humidité absolue et la température. Cet effet est masqué dans la région tempérée par leur forte corrélation entre eux.
\end{itemize}

\section{Perspectives}

Les relations entre l'incidence de la dengue et les facteurs environnementaux du 64 provinces au Vietnam a été analysée très détails en utilisant des différentes méthodes. Cependant, nous devons approfondir l'analyse des données des provinces d'autres pays dans la région Asie du Sud-Est pour obtenir un aperçu plus complet et détaillé de l'influence des facteurs environnementaux sur l'évolution de l'épidémie de la dengue. 

Les connaissance sur l'influence des facteurs environnementaux et l'épidémie nous permettent de construit un système d'alerte visant à renforcer les mesures de prévention de l'épidémie dans les périodes où l'éclosion de l'épidémie est venu (Par exemple : au début la saison de la pluie où le température et l'absolue humidité fortement augmentent). 

De plus, nous pouvons développer un logiciel en utilisant les méthode d'analyse dans cette thèse pour aider les utilisateur ordinaire à decomposer les relations entres des séries temporelles. 



