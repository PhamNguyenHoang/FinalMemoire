
%\geometry
\usepackage[a4paper]{geometry}
\geometry{left=3cm,right=2.5cm,top=2.5cm,bottom=2.5cm} % mt choice

% d�finir le style des textes et paragraphes
\usepackage[utf8]{inputenc}
\usepackage[T1]{fontenc}
\usepackage{enumerate}

% for paragraphe spacing
\usepackage{ragged2e} % for paragraphe spacing and justifying of the whole document

\usepackage{lipsum} %Pour faire des essais

% Choix de la langue
\usepackage[english,francais]{babel}

% Lmodern et substitution des petites capitales grasses manquantes
\usepackage{lmodern}
\rmfamily
\DeclareFontShape{T1}{lmr}{b}{sc}{<->ssub*cmr/bx/sc}{}
\DeclareFontShape{T1}{lmr}{bx}{sc}{<->ssub*cmr/bx/sc}{}


% Diff�rents paquets pour les maths
\usepackage{amssymb, amsmath, amsthm, amscd}
\usepackage{mathrsfs}

% Pour les figures
\usepackage{subfig}

%Pour la page de garde
\usepackage{tabularx} % Permet d'utiliser l'environnement tabularx
\usepackage{calc} % Pour pouvoir donner des formules dans les d�finitions de longueur
\usepackage{graphicx} % Pour inclure des graphiques 
% Attention : pour inclure des .jpg comme dans l'exemple (ou des .png ou .pdf)
% il faut compiler directement en pdf (commande pdflatex).
% Pour inclure des .eps, il faut compiler avec latex + dvips + ps2pdf.

% Pour avoir des liens hypertexte dans le document compil�
\usepackage{hyperref}
%\usepackage{nohyperref} % � utiliser pour pouvoir compiler sans g�n�rer des liens

% Pour mettre la bibliographie dans la table des mati�res avec le bon num�ro de page (voir plus loin)
\usepackage[nottoc,notlof,notlot]{tocbibind}

% Table de mati�res pour les chapitres 
\usepackage[french]{minitoc}
\setcounter{minitocdepth}{2}
\mtcindent=15pt % Use \minitoc where to put a table of contents

% Pour l'index des notations
\usepackage{makeidx}

\usepackage{epstopdf}

\makeindex